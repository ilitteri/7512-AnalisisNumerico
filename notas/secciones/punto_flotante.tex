\documentclass[../main.tex]{subfiles}

\subsection{Ejemplos disparadores}
\begin{itemize}
    \item 
        \begin{lstlisting}[language=Python]
        0.25 - 0.5 - 0.5 - 0.5 - 0.5 - 0.5
        0
        \end{lstlisting}
    \item 
        \begin{lstlisting}[language=Python]
        0.5 - 0.1 - 0.1 - 0.1 - 0.1 - 0.1
        2.7755575615628914e-17
        \end{lstlisting}
    \item 
        \begin{lstlisting}[language=Python]
        (1 + 1e-15) == 1
        False
        \end{lstlisting}
    \item 
        \begin{lstlisting}[language=Python]
        (1 + 1e-16) == 1
        True
        \end{lstlisting}
    \item
        \begin{lstlisting}[language=Python]
        a = 1/3e12
        b = 1
        c = -1
        a + b + c == a + (b + c)
        False
        \end{lstlisting}
\end{itemize}

\subsection{Números de máquina}

Todo número de máquina van a estar simbolizados por

\begin{equation*}
    \mathcal{M} = \left\{m: m = (-1)^{S} \cdot C \cdot 2^{Q}, S, b_{i} \in \{0, 1\}, C = 1.b_{1}b_{2}\dots b_{p}, E_{min} \leq Q \leq E_{max} \right\}
\end{equation*}

en donde $S$ es el \textit{bit de signo} y corresponde a un único bit, $C$ es la \textit{mantisa} o \textit{significado} a la que se le asigna una cantidad de bits, que se llama precisión, $Q$ es el \textit{exponente}.

\subsection{Propiedades destacables}

\begin{itemize}
    \item Los números de máquina son un subconjunto finito de los números racionales.
    \item Los números de máquina son más densos a valores absolutos menores, y se separan más a medida que se alejan del 0.
    \item La cota para el error relativo es constante para cualquier número distinto de 0 que se represente.
\end{itemize}

El mayor (valor absoluto) número de máquina que podemos formar es $nmax = (1.1111\dots)\cdot2^{E_{max}}$. Ejemplo: $E_{max} = 1023$. Eso da aproximadamente $1.8^{308}$. Al intervalo [-nmax, nmax] se lo conoce como rango de máquina.

Si intentamos almacenar un número más grande en valor absoluto que $nmax$ obtendremos una excepción llamada \textit{overflow}.

La computadora devuelve el resultada infinito (con signo) y debe levantar una advertencia en la ejecución del programa.

El menor (en valor absoluto) número de máquina que podemos formar es $nmin = (1.1111\dots)\cdot2^{E_{min}}$. Ejemplo: $E_{min} = -1022$. Eso da aproximadamente $9.8^{-324}$. El intervalo [-nmin, nmin]es un hueco alrededor del cero. 

Si intentamos almacenar un número más chico en valor absoluto que $nmin$ obtendremos una excepción llamada \textit{underflow}.

La computadora devuelve el resultado cero (con signo) y debe levantar una advertencia en la ejecución del programa.

Cuando queremos almacenar un número cualquiera, la computadora lo redeondea a un número de máquina. El criterio más habitual es redondear al más cercano.

TABLA(screenshot )   

\subsection{Formatos típicos}
\subsection{Aritmética de punto flotante}
\subsection{Ejemplo final}